%% %%%%%%%%%%%%%%%%%%%%%%%%%%%%%%%%%%%%%%%%%%%%%%%%
%% Problem Set/Assignment Template to be used by the
%% Food and Resource Economics Department - IFAS
%% University of Florida's graduates.
%% %%%%%%%%%%%%%%%%%%%%%%%%%%%%%%%%%%%%%%%%%%%%%%%%
%% Version 1.0 - November 2019
%% %%%%%%%%%%%%%%%%%%%%%%%%%%%%%%%%%%%%%%%%%%%%%%%%
%% Ariel Soto-Caro
%%  - asotocaro@ufl.edu
%%  - arielsotocaro@gmail.com
%% %%%%%%%%%%%%%%%%%%%%%%%%%%%%%%%%%%%%%%%%%%%%%%%%
%Numbered environment




\documentclass[12pt]{article}
\usepackage{design_ASC}
\theoremstyle{definition}
\newtheorem{exmp}{Example}[section]
\newtheorem{slo}{Solution}[section]
\newcommand*{\Perm}[2]{{}^{#1}\!P_{#2}}%
\newcommand*{\Comb}[2]{{}^{#1}C_{#2}}%
%% -----------------------------
%% TITLE
%% -----------------------------
\title{\textbf{Lecture 3}} %% Assignment Title

\author{\textbf{Ibrahim} Abou Elenein}

\date{\today} %% Change "\today" by another date manually
%% -----------------------------
%% -----------------------------

%% %%%%%%%%%%%%%%%%%%%%%%%%%
\begin{document}
\setlength{\droptitle}{-5em}    
%% %%%%%%%%%%%%%%%%%%%%%%%%%
\maketitle
% --------------------------
% Start here
% --------------------------

% %%%%%%%%%%%%%%%%%%%
\section{Sample Spaces}
The set of all possible outcomes of a random
experiment is called a \textit{sample space}
\section{Events}
An event E, is a set of some outcomes of a
probability experiment, i.e. a subset of the sample
\[
    E \subseteq S
\]
If an event E contains no outcomes, then E is an impossible event.

\section{Probability}
\[
    P(E) = \frac{N(E)}{N(S)}
\]
\[
    0 \leq P (E) \leq 1  ;  \ P(S) = 1 
\]
\begin{exmp}
    A card is drawn from a standard deck.
    Find the probabilities of the following events.
    \begin{itemize}
        \item Getting a queen.
        \item Getting a club.
        \item Getting a number.
    \end{itemize}       

\end{exmp}    
\begin{slo}
    \[
        |S| = 52
    \]
    E1 is the event of getting a queen,
    $\displaystyle P(E1) = \frac{|E1|}{|S|} = \frac{4}{52}=  \frac{1}{13}$
    \\ 
    E2 is the event of getting a club,
    $\displaystyle P(E2) = \frac{|E2|}{|S|} = \frac{13}{52}=  \frac{1}{4}$ \\
    E3 is the event of getting a number,
    $ \displaystyle P(E3) = \frac{|E3|}{|S|} = \frac{40}{52}=  \frac{10}{13}$ \\


\end{slo}

\section{Counting Rules}
\subsection{Multiplication Rule}
In a sequence of two experiments, if the
first experiment can occur in $m$ different
ways and the second one can occur in $n$
different ways, then the whole sequence
can occur in $m \times n$  different ways.
\begin{exmp}

    Consider the manufacturing of:
    number-plates consisting of two letters followed by four digits
    \begin{enumerate}
        \item  How many plates are possible?
            \[
                26 \times 26 \times 10 \times 10 \times 10 \times 10 = 6760000
            \]
        \item How many plates are possible, if no letter or digit can be repeated?
            \[
                26 \times 25 \times 10 \times 9 \times 8 \times 7 = 3276000 
            \]
    \end{enumerate}

\end{exmp}

\begin{exmp}
    In a class of 10 students, 6 are to be chosen and seated in a row for a
    picture.
    How many different pictures are possible?
    \[
        10 \times 9 \times 8 \times 7 \times 6 \times 5 \ different \  pictures.
    \]
    \textbf{Tip} what if they were to sit in a circle?
\end{exmp}

\subsection{Permutaion}
An arrangement of $n$ objects \textsc{ a specific order} using $k$ objects at a time is
called a permutation and it is denoted by $\Perm{n}{k}$. 
The number of repetition-free
permutations (linear arrangements) of size $k$ from a set of $n$ distinct objects
is given by:
\[
    \Perm{n}{k} = \frac{n!}{(n - k)!}; \ \ 0 \leq k \leq n
\]

\subsection{Combination}
A selection of $k$ distinct objects \textsc{without regard to order} out of $n$
objects is called a combination and it is denoted by $\Comb{n}{k} $
The number of repetition-free combinations of size k from n distinct objects is given:
\[
    \Comb{n}{k} = \frac{\Perm{n}{k}}{k!} = \frac{n!}{(n - k)!k!}; \ \  0 \leq k \leq n
\]
\subsection{Problems}
\begin{exmp}

\end{exmp}    
\begin{exmp}
    In a class of 10 students, three are to be chosen to represent the class in a
    competition.  How many selections are possible? \\

    Note that, here, the students are not selected in any 
    specific order.

    The number of selections is $\Comb{10}{3} $

\end{exmp}    

\begin{exmp}
    A student is taking a Math-401 test in which 7 questions out of 10 are to be
    answered. In how many ways can the student answer the exam if :
\end{exmp}    
\begin{enumerate}
    \item Any 7 questions may be selected.
        \begin{center}
            The student can answer the exam in $\Comb{10}{7}$
        \end{center}
    \item The first 2 questions must be selected.
        \begin{center}
            The student can answer the exam in $\Comb{8}{5}$
        \end{center}
    \item The student must choose 3 questions from the first 5 and 4 questions from the
        \begin{center}
            The student can answer the exam in $\Comb{5}{3} \times \Comb{5}{4} = 50$
        \end{center}
\end{enumerate}       


\begin{exmp}
    What is the number of (possibly meaningless) words that are made up of all the
    letters in the word \textbf{chemistry}
\end{exmp}    
\begin{center}
    The number of such words id $\Perm{9}{9}$
\end{center}

\begin{exmp}
    What is the number of permutations of the letters in the word \textbf{ball}?
    \begin{center}
        Note that there are repeated letter so it's not 4 distinct objects.
    \end{center}    
    \begin{center}
        The number of permutations is $\displaystyle  \frac{\Perm{4}{4}}{2!} = 12$
    \end{center}    
\end{exmp}    

\begin{exmp}
    What is the number of permutations of the letters in the word \textbf{Pepper}?

    \begin{center}
        The number of permutations is $\displaystyle  \frac{\Perm{6}{6}}{2!3!}$
    \end{center}   
\end{exmp}    

\begin{exmp}
    The owner of a pizzeria prepares every pizza by always combining 4 different
    ingredients. How many ingredients does he need, at least, if he would like to
    offer 30 different pizzas in the menu?
    \begin{center}
        we need to find $n$ such that $\Comb{n}{4} = 30$ 
    \end{center}    
    \begin{center}
        $\displaystyle \frac{n!}{(n-4)!4!} = 30$ \\
    \end{center}    
    \begin{center}
        $\displaystyle \frac{n(n-1)(n-2)(n-3)(n-4)!}{(n-4)!} = 30 \times 4!$
    \end{center}    
    \begin{center}
        $\displaystyle n(n-1)(n-2)(n-3) - 720 = 0$ \\

        By solving the equation $n = 6.8 \Rightarrow $he needs at least 7 ingredients
    \end{center}    

\end{exmp}    

\begin{exmp}
    You play a simple card game. You draw 3 cards. If any of them is a King or
    the three cards are of the same suit, you win otherwise you lose.
    How many hands are you winning? 
    \begin{center}
        A = draw at least one king in a hand of 3 cards \\
        N(A) = One King or Two or Three = \\
        $ (\Comb{4}{1} \times \Comb{48}{2})+ (\Comb{4}{2} \times \Comb{48}{1})
        + (\Comb{4}{3} \times \Comb{48}{0})
        $
    \end{center}    
    \begin{center}
        B = A hand draw 3 cards with  same suit \\
        $N(B) = \Perm{4}{1} \times \Perm{13}{3}$ \\
    \end{center}   
    \begin{center}
        $A \cap B$ = a hand with draw 3 cards with a king and the same suit.
        $N(A \cap B) = \Comb{4}{1} \times \Comb{1}{1} \Comb{12}{2}  $
    \end{center}   
    \begin{center}
        $N(A \cup  B) = N(A) + N(B) - N(A \cap B) = 5684.$    
    \end{center}
\end{exmp}    

\begin{exmp}
    In how many ways you can arrange the word “ORANGE”, if: 
    \begin{itemize}
        \item 2 vowels and 2 consonants are used to make 4-letter words.  
            \begin{center}
                No.of ways = $\Comb{3}{2} \times \Comb{3}{2} \times 4! $
            \end{center}   
        \item 2 vowels and 3 consonants are used to make 5-letter words. 
            \begin{center}
                No.of ways = $\Comb{3}{2} \times \Comb{3}{3} \times 5! $
            \end{center}   
        \end{itemize}
\end{exmp}
\begin{exmp}
    There are 8 men and 9 women on a committee selection pool.
    A committee consisting of:
    a president, a vice-president, and 3 coordinators; is to
    be formed.  
    In how many ways can exactly three women be on the committee?
    \begin{center}
        $\displaystyle N = \Comb{9}{3} \times \Comb{8}{2}  \times [\frac{5!}{3!}]$  
    \end{center}

\end{exmp}    
\end{document}
